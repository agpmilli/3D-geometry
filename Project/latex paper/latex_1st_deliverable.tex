\documentclass[10pt,conference,compsocconf]{IEEEtran}

\usepackage{hyperref}
\usepackage{graphicx}	% For figure environment


\begin{document}
\title{Digital 3D Geometry Processing - Project}

\author{
  Raphaël Steinmann\\
	Thomas Batschelet\\
	Alain Milliet\\
  \textit{Department of Computer Science, EPFL, Switzerland}
}

\maketitle

\begin{abstract}
  TODO
\end{abstract}

\section{Introduction}
TODO


\section{Tips for Good Writing}
\label{sec:tips-writing}


\subsection{Abstract}
\label{subsec:Abstract}

\begin{enumerate}
\item Example
\end{enumerate}

\subsection{Figures and Tables}

%\begin{figure}[tbp]
%  \centering
%  \includegraphics[width=\columnwidth]{denoised_signal_1d}
%  \caption{Signal compression and denoising using the Fourier basis.}
%  \vspace{-3mm}
%  \label{fig:denoise-fourier}
%\end{figure}
%\begin{figure}[htbp]
%  \centering
%  \includegraphics[width=\columnwidth]{local_wdenoised_1d}
%  \vspace{-3mm}
%  \caption{Signal compression and denoising using the Daubechies wavelet basis.}
%  \label{fig:denoise-wavelet}
%\end{figure}

Use examples and illustrations to clarify ideas and results. For
example, by comparing Figure~\ref{fig:denoise-fourier} and
Figure~\ref{fig:denoise-wavelet}, we can see the two different
situations where Fourier and wavelet basis perform well. 

\subsubsection{Equations}

There are three types of equations available: inline equations, for
example $y=mx + c$, which appear in the text, unnumbered equations
$$y=mx + c,$$
which are presented on a line on its own, and numbered equations
\begin{equation}
  \label{eq:linear}
  y = mx + c
\end{equation}
which you can refer to at a later point (Equation~(\ref{eq:linear})).

\subsubsection{Tables and Figures}

Tables and figures are ``floating'' objects, which means that the text
can flow around it.
Note
that \texttt{figure*} and \texttt{table*} cause the corresponding
figure or table to span both columns.



\section{Summary}
TODO


\end{document}
